


\documentclass[border=9,tikz]{standalone}
\begin{document}
\def\LennaIpsum{\includegraphics[width=2cm]{lenna.png}\llap\LaTeX}
\def\RememberInversion(#1,#2){
	\expandafter\xdef\csname Inv(\u,\v)x\endcsname{\xx}
	\expandafter\xdef\csname Inv(\u,\v)y\endcsname{\yy}
}
\def\RecallInversion#1(#2,#3){
	\expandafter\xdef\csname#1x\endcsname{\csname Inv(#2,#3)x\endcsname}
	\expandafter\xdef\csname#1y\endcsname{\csname Inv(#2,#3)y\endcsname}
}
\tikz{
	\draw circle(.05)circle(5)(3,0)node{\LennaIpsum}(12.5,0);
	\draw[dotted](4,-5)--(4,5)(3.125,0)circle(3.125);
	\foreach\u in{-10,...,10}{
		\foreach\v in{-10,...,10}{
			% Affine transformation of (u, v), unit mm
			\pgfmathsetmacro\uu{\u+30}
			\pgfmathsetmacro\vv{\v+0}
			\pgfmathsetmacro\rr{\uu*\uu+\vv*\vv}
			% Take inversion, unit mm
			\pgfmathsetmacro\xx{50*\uu/\rr*50}
			\pgfmathsetmacro\yy{50*\vv/\rr*50}
			% Remember the coordinates
			\RememberInversion(\u,\v)
		}
	}
	\foreach\u in{-10,...,9}{
		\foreach\v in{-10,...,9}{
			% For every square, recall the coordinates of the four corners
			\pgfmathtruncatemacro\U{\u+1}
			\pgfmathtruncatemacro\V{\v+1}
			\RecallInversion NW(\u,\V)	\RecallInversion NE(\U,\V)
			\RecallInversion SW(\u,\v)	\RecallInversion SE(\U,\v)
			% The lower left triangle ◺
			\pgfmathsetmacro\aa{\SEx-\SWx}	\pgfmathsetmacro\ab{\SEy-\SWy}
			\pgfmathsetmacro\ba{\NWx-\SWx}	\pgfmathsetmacro\bb{\NWy-\SWy}
			\pgflowlevelobj{
				\pgfsettransformentries\aa\ab\ba\bb{\SWx mm}{\SWy mm}
			}{
				\clip(1.02mm,0)-|(0,1.02mm);
				\path(-\u mm,-\v mm)node{\LennaIpsum};
			}
			% The upper right triangle ◹
			\pgfmathsetmacro\aa{\NEx-\NWx}	\pgfmathsetmacro\ab{\NEy-\NWy}
			\pgfmathsetmacro\ba{\NEx-\SEx}	\pgfmathsetmacro\bb{\NEy-\SEy}
			\pgflowlevelobj{
				\pgfsettransformentries\aa\ab\ba\bb{\NEx mm}{\NEy mm}
			}{
				\clip(-1mm,.01mm)-|(.01mm,-1mm);
				\path(-\U mm,-\V mm)node{\LennaIpsum};
			}
		}
	}
}
\end{document}


