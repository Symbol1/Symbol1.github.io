% !TEX encoding = UTF-8 Unicode
% !TEX program = lualatex

\documentclass{article}

\usepackage[margin=70pt]{geometry}
    \parindent0pt
    \parskip10pt plus10pt

\usepackage{fontspec, pgfmath, xcolor, xurl}
    \setmainfont{NotoSerif}[ItalicFont=*-Italic, BoldFont=*-Bold]
    \setsansfont{NotoSerif}%{NotoSans}
    \setmonofont{NotoSerif}%{NotoSansMono}
    \def\texttc#1{{\fontspec{NotoSerifTC}\raisebox{-0.07em}{#1}}}
    \def\sec#1{\vskip1em\textbf{\fs1#1}}
    \def\fs#1{%
        \pgfmathsetmacro\a{#1}%
        \pgfmathsetmacro\A{\parskip*(4/3)^\a}%
        \pgfmathsetmacro\B{\A*(4/3)}%
        \fontsize{\A pt}{\B pt}\selectfont%
    }

\usepackage[colorlinks, allcolors=blue!40!black]{hyperref}
    \def\email#1{\href{mailto:#1}{#1}}

\begin{document}

\leftskip0pt plus1fill
\rightskip0pt plus1fill

\fs2
                                          CURRICULUM VITAE

                                 Hsin-Po Wang \kern2em \texttc{王新博}

\fs0

            Google Scholar: \url{https://scholar.google.com/citations?user=tJ8-ChgAAAAJ}

          Website: \url{https://www.hsin-po.wang} \hfill Email: \email{hsinpo@ntu.edu.tw}

\leftskip0pt
\rightskip0pt
\fs0

\sec{Education}

Ph.D. in Math \hfill       University of Illinois Urbana-Champaign, Illinois       \hfill 2016--2021

B.Sc. in Math \hfill    National Taiwan University \texttc{國立臺灣大學}, Taiwan   \hfill 2011--2015

\sec{Employment}


Assistant Professor                                                             \hfill August 2024--
\\ Department of Electrical Engineering \texttc{電機工程學系}
\\ Graduate Institute of Communication Engineering \texttc{電信工程學研究所}
\\ National Taiwan University \texttc{國立臺灣大學}, Taiwan

Apple Research Fellow                                                       \hfill January--May 2024
\\ Simons Institute for the Theory of Computing, California

Postdoctoral Scholar                                              \hfill October 2022--December 2023
\\ Department of Electrical Engineering and Computer Sciences,
\\ University of California, Berkeley, California

Postdoctoral Scholar                                             \hfill October 2021--September 2022
\\ Department of Electrical and Computer Engineering,
\\ University of California San Diego, California

Teaching Assistant                                                   \hfill September 2016--May 2023
\\ Department of Mathematics,
\\ University of Illinois Urbana-Champaign, Illinois

\sec{Awards and Honors}

Irving Reiner Memorial Award in Algebra                                                  \hfill 2021

Research Assistant Fellowship                                                     \hfill Spring 2020

Teacher ranked as excellent by their students             \hfill Fall 2019, Spring 2019, Spring 2018

Dean's List \#2 when graduation \texttc{理學院院長獎}                                    \hfill 2015

top 5\% GPA \texttc{書卷獎}        \hfill Fall 15, Spring 14, Spring 13, Fall 12, Spring 12, Fall 11

Prof. Cheng-Tang Hsiao Memorial Scholarship \texttc{蕭正堂紀念獎學金}                    \hfill 2014

Prof. Ta-Kai Hu Memorial Scholarship \texttc{胡達開先生紀念獎學金}                       \hfill 2013

\sec{Research Interests}

\def\interest#1{\textcolor{red!20!black}{#1}}
\def\work#1{\textcolor{yellow!30!black}{#1}}
\def\apply#1{\textcolor{green!20!black}{#1}}

My interest are in
\interest{information theory} and
\interest{coding theory}.
Working on
\work{polar codes} (wireless communication), 
\work{group testing} (with many downstream applications),
\work{regenerating codes} (cloud storage),
\work{distributed matrix multiplication} (cloud computation), and
\work{DNA digital data storage} (archival storage),
I specialize in finding applications of
\apply{algebra},
\apply{combinatorics},
\apply{calculus},
\apply{probability theory},
and other mathematical tools to said topics.

\bgroup
\def\section#1#2{\sec{Journal Publications \mdseries (new to old)}}
\begin{thebibliography}{J9}
    \bibitem[J6]{J6}
    H.-P. Wang and R. Gabrys and A. Vardy.
    \href{https://doi.org/10.1109/TIT.2023.3282847}
    {Tropical Group Testing}.
    \emph{IEEE Transactions on Information Theory}.
    June 2023.

    \bibitem[J5]{J5}
    H.-P. Wang, T.-C. Lin, A. Vardy, R. Gabrys.
    \href{https://doi.org/10.1109/TIT.2023.3253074}
    {Sub-4.7 Scaling Exponent of Polar Codes}.
    \emph{IEEE Transactions on Information Theory}.
    March 2023.

    \bibitem[J4]{J4}
    I. Duursma, H.-P. Wang.
    \href{https://doi.org/10.1007/s00200-021-00526-3}
    {Multilinear Algebra for Minimum Storage Regenerating Codes: A Generalization of Product-Matrix Construction}.
    \emph{Applicable Algebra in Engineering, Communication and Computing}.
    October 2021.

    \bibitem[J3]{J3}
    I. Duursma, X. Li, H.-P. Wang.
    \href{https://doi.org/10.1137/20M1346742}
    {Multilinear Algebra for Distributed Storage}.
    \emph{SIAM Journal on Applied Algebra and Geometry (SIAGA)}.
    September 2021.

    \bibitem[J2]{J2}
    H.-P. Wang, I. Duursma.
    \href{https://doi.org/10.1109/TIT.2020.3041523}
    {Log-logarithmic Time Pruned Polar Coding}.
    \emph{IEEE Transactions on Information Theory}.
    March 2021.

    \bibitem[J1]{J1}
    H.-P. Wang, I. Duursma.
    \href{https://doi.org/10.1109/TIT.2020.3041570}
    {Polar Codes' Simplicity, Random Codes' Durability}.
    \emph{IEEE Transactions on Information Theory}.
    March 2021.
\end{thebibliography}
\egroup

\bgroup
\def\section#1#2{\sec{Peer-Reviewed Conference Publications \mdseries (new to old)}}
\begin{thebibliography}{C99}
    \bibitem[C11]{C11}
    V. Guruswami, H.-P. Wang.
    \href{https://doi.org/10.4230/LIPIcs.APPROX/RANDOM.2024.65}
    {Capacity-Achieving Gray Codes}.
    \emph{International Conference on Randomization and Computation (RANDOM)}.
    August 2024.

    \bibitem[C10]{C10}
    H.-P. Wang, C.-W. Chin.
    \href{https://doi.org/10.1109/ISIT57864.2024.10619178}
    {On Counting Subsequences and Higher-Order Fibonacci Numbers}.
    \emph{IEEE International Symposium on Information Theory (ISIT)}.
    July 2024.
 
    \bibitem[C9]{C9}
    H.-P. Wang, V. Guruswami.
    \href{https://doi.org/10.1109/ISIT57864.2024.10619251}
    {Successive Cancellation Sampling Decoder: An Attempt to Analyze List Decoding Theoretically}.
    \emph{IEEE International Symposium on Information Theory (ISIT)}.
    July 2024.
 
    \bibitem[C8]{C8}
    H.-P. Wang, V. Guruswami.
    \href{https://doi.org/10.1109/ISIT57864.2024.10619098}
    {Isolate and then Identify: Rethinking Adaptive Group Testing}.
    \emph{IEEE International Symposium on Information Theory (ISIT)}.
    July 2024.

    \bibitem[C7]{C7}
    H.-P. Wang, R. Gabrys, V. Guruswami.
    \href{https://doi.org/10.1109/ISIT54713.2023.10206843}
    {Quickly-Decodable Group Testing with Fewer Tests: Price-Scarlett's Nonadaptive Splitting with Explicit Scalars}.
    \emph{IEEE International Symposium on Information Theory (ISIT)}.
    June 2023.

    \bibitem[C6]{C6}
    H.-P. Wang, C.-W. Chin.
    \href{https://doi.org/10.1109/ISIT54713.2023.10206540}
    {Density Devolution for Ordering Synthetic Channels}.
    \emph{IEEE International Symposium on Information Theory (ISIT)}.
    June 2023.

    \bibitem[C5]{C5}
    T.-C. Lin, H.-P. Wang.
    \href{https://doi.org/10.1109/ISIT54713.2023.10206451}
    {Optimal Self-Dual Inequalities to Order Polarized BECs}.
    \emph{IEEE International Symposium on Information Theory (ISIT)}.
    June 2023.

    \bibitem[C4]{C4}
    H.-P. Wang, V. Guruswami.
    \href{https://doi.org/10.1109/ISIT54713.2023.10206989}
    {How Many Matrices Should I Prepare to Polarize Channels Optimally Fast?}
    \emph{IEEE International Symposium on Information Theory (ISIT)}.
    June 2023.

    \bibitem[C3]{C3}
    H.-P. Wang, V.-F. Dragoi.
    \href{https://doi.org/10.1109/ISIT54713.2023.10206704}
    {Fast Methods for Ranking Synthetic BECs}.
    \emph{IEEE International Symposium on Information Theory (ISIT)}.
    June 2023.

    \bibitem[C2]{C2}
    I. Duursma, R. Gabrys, V. Guruswami, T.-C. Lin, H.-P. Wang.
    \href{https://doi.org/10.4230/LIPIcs.APPROX/RANDOM.2022.17}
    {Accelerating Polarization via Alphabet Extension}.
    \emph{International Conference on Randomization and Computation (RANDOM)}.
    September 2022.

    \bibitem[C1]{C1}
    H.-P. Wang, R. Gabrys, A. Vardy.
    \href{https://doi.org/10.1109/ISIT50566.2022.9834718}
    {PCR, Tropical Arithmetic, and Group Testing}.
    \emph{IEEE International Symposium on Information Theory (ISIT)}.
    June 2022.
\end{thebibliography}
\egroup

\bgroup
\def\section#1#2{\sec{Invited Talks \mdseries (new to old)}}
\begin{thebibliography}{T9}
    \bibitem[T5]{T5}
    \href{(https://umi.dm.unibo.it/jm-umi-ams/special-sessions/special-sessions-on-25-26-july-2024/)}
    {How to Speak Tensor}.
    \emph{International AMS-UMI Joint Meeting}.
    July 2024, Palermo, Italy.

    \bibitem[T4]{T4}
    \href{https://www.ce.cit.tum.de/en/lnt/events/2024-coding-theory-and-algorithms-for-dna-based-data-storage/program/}
    {Geno-Weaving: Low-Complexity Capacity-Achieving Data Storage on DNA}.
    \emph{Coding Theory and Algorithms for DNA-based Data Storage (ISIT2024 Satellite Workshop)}.
    July 2024, Athens, Greece.

    \bibitem[T3]{T3}
    \href{https://ita.ucsd.edu/workshop/schedule}
    {GenoWeave: Interleaving Polar Codes Across Strands for DNA Data Storage}.
    \emph{Information Theory and Applications Workshop (ITA)}.
    February 2024, San Diego, California.

    \bibitem[T2]{T2}
    \href{https://meetings.ams.org/math/jmm2024/meetingapp.cgi/Paper/29146}
    {Channel Manipulation as a Coding Technique}.
    \emph{Joint Mathematics Meetings (JMM)}.
    January 2024, San Francisco, California.

    \bibitem[T1]{T1}
    \href{https://meetings.siam.org/sess/dsp_programsess.cfm?SESSIONCODE=72368}
    {Moulin Codes}.
    \emph{SIAM Conference on Applied Algebraic Geometry (AG21)}.
    August 2021, virtual.
\end{thebibliography}
\egroup

\end{document}
